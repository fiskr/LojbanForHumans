\documentclass[12pt]{book}
\usepackage{hyperref}
\usepackage{fixltx2e} % for \textsubscript{} and \textsuperscript{}, needed for 2015 or later

\begin{document}

\title{Lojban for Humans}

\author{la finpe}


\maketitle

\chapter*{Preface}

The title of this book, \emph{Lojban for Humans}, has several meanings. In particular, it is meant as a tongue-in-cheek reference to how cryptic some language-learning texts can be. With regards to Lojban, there is arguably \emph{one}, main book: \textit{The Complete Lojban Language} (CLL). \cite{CLL} The CLL is an excellent book with plenty of value for learning, but it tries to fulfill multiple roles -- as a tool for teaching Lojban, and as a complete description of the Lojban language. Since it functions as a specification for a language, it is forced to take a structure that can frustrate even the dedicated student.

This book is meant to be for Lojban what \textit{Wheelock's Latin} \cite{WheelocksLatin} was for Latin. It is meant to bridge the gap between beginners and experts. As such, this book strives to be, as Wheelock said of his own book, ``a beginner's book which is mature, humanistic, challenging, and instructive, which at the same time is reasonable in its demands.'' \cite[Preface, p.~xiii]{WheelocksLatin}

In addition to teaching Lojban, this books aims to teach the basic concepts necessary to \emph{understand} language. Just as \textit{Wheelock's Latin} often teaches English speakers for the first time basic concepts of language, such as grammatical cases, this book aims to elevate the general language knowledge of the reader. Learning Lojban is a great opportunity for exploring language itself, and where possible the reader will be given the introduction to new concepts as needed, and references to where they could learn more if they were so inclined.  

Additionally, this book will focus on providing examples and building up vocabulary as grammatical concepts are introduced. To separate this work from the already thorough, deductive approach \textit{The Complete Lojban Language} takes, this book will give readers words to play with and interesting texts to translate and use. As with any language, practice is extremely important, and as such it will form a core part of this book.  


%%%%%%%%%%%%%%%%%%%%%%%%%%%%%%%%%%%%%%%%%%%%%%%%%%%
%%%%%%%%%%%%%%%%%%%%%%%%%%%%%%%%%%%%%%%%%%%%%%%%%%%
%%%%%%%%%%%%%%%%%%%%%%%%%%%%%%%%%%%%%%%%%%%%%%%%%%%
% End of Chapter

\tableofcontents

\chapter{Introduction}

ni'o ni'o no mo'o

\section{A Brief History of Lojban}

Lojban is a constructed language. This means its structure is artificial -- designed with a top-down approach, like how a building is architected or a law is drafted. This differs from natural languages, which evolved organically, like DNA. \footnote{Wikipedia has an \href{https://en.wikipedia.org/wiki/Top-down_and_bottom-up_design}{article} about top-down and bottom-up design which may give more context for this distinction.}

% Steven Pinker describes language as a "grass-roots phenomenon" - let's try to find a book that connects to this concept for readers to follow up on
% I also worry that the distinction about being artificial and natural seems arbitrary, and the views about whether language was discontinuous or continuous are controversial, so I'm not sure it's fair to claim it's bottom-up entirely, either (though it would be hard to argue it's not a bottom up phenomenon after the first inklings of language came about - vocabulary and other structures to language certainly seem to be correlated geographically and created through and maintained by culture). 


Work to create Lojban was started in 1987 by the Logical Language Group. It was based on a linguistic project that produced a constructed language called \emph{Loglan}. The creator of Loglan wanted to keep the language proprietary and did not want other people to use it openly. As a result a group of people created Lojban in the spirit of Loglan but open and free.  

\subsection{Loglan}

Loglan was created in 1955 by James Cooke Brown. He created the language to conduct linguistic research to investigate the Sapir-Whorf hypothesis.

The \textbf{Sapir-Whorf hypothesis}, also known as \emph{the principle of linguistic relativity}, is the idea that one's language impacts one's way of thinking, or one's perception of the world.

The goal of the research was to test whether an artificial language could be designed that was so different from natural languages that speakers of this new language would think differently, thus testing the Sapir-Whorf hypothesis. 

To develop Loglan, Brown created \textbf{The Loglan Institute} (TLI). He wrote many papers about the design of Loglan, but the project was never considered complete. Because Brown claimed legal restrictions on the use of Loglan, the Logical Language Group was formed by followers of Brown and they split away from The Loglan Institute. 

Though Brown claimed legal restriction on the language, the United States Patent and Trademark Office eventually ruled that Loglan was not able to be restricted the ways Brown claimed. However, this ruling came after the Logical Language Group had split from The Loglan Institute, and by then Loglan had been abandoned for a new language: Lojban. 
%TODO: Cite these claims!


%%%%%%%%%%%%%%%%%%%%%%%%%%%%%%%%%%%%%%%%%%%%%%%%%%%
% End of Subsection

\subsubsection{The Design Goals of Loglan}

The direct goal of Loglan was to test whether a logical language could cause people speaking that language to think more logically. 

The grammar of Loglan, thus, is based on predicate logic -- in the spirit of Leibniz's \emph{universal symbolism}, by which Leibniz predicted ``the rational powers of man would be marvelously extended.'' \cite{BrownSci} 

It was designed to be easy to teach and use, but sophisticated enough to let people effectively use the language as they would a natural language.

The language was also designed to be culturally neutral. 
Despite this, words are constructed from elements of semantically related words from the eight most widely spoken languages on Earth, to make the language easier to learn. \footnote{Being based on another language makes Loglan an \textit{a posteriori} language. See \href{https://en.wikipedia.org/wiki/Constructed_language\#a_posteriori_language}{this Wikipedia article} for more on constructed, \textit{a posteriori} languages.}


It was also meant to require of the speaker as few obligatory categories as possible. For example, in English it is impossible to express a finite verb without expressing the tense of that verb, making the time tense a obligatory category. Loglan attempted to avoid these obligatory categories as much as possible. 

% TODO: Expand upon obligatory categories and metaphysical parsimony

The language was designed to be parsed in only one way and to be unambiguous. 

The syllabic structure of words was designed so that a sequence of syllables could be separated into words in only one way, even if not clear from pauses in speech. 

The language has a small number of phonemes so that regional accents would be less likely to render speech unintelligible. 


\subsubsection{The Importance of Unambiguous Language}


Steven Pinker illustrates the importance of unambiguous language in a lecture of his. On one hand, we have a rather non-controversial statement: 

\begin{quote}
Conan discusses \textbf{sex} \\
with Dr. Ruth
\end{quote}

In this quote, Conan is going to discuss with his guest the topic of sex (the topic is in bold). However, this sentence is ambiguous in its current form, and could be interpreted in an unintended and embarrassing way:
 
\begin{quote}
Conan discusses \\
\textbf{sex with Dr. Ruth}
\end{quote}

Now the sentence seems to state that Conan is going to discuss the topic of \emph{sex with Dr. Ruth}, something likely not intended as the original meaning, but not an impossible interpretation nonetheless. \cite[22:58]{PinkerWindow}

% Also: cite Freakonomics episode about the cost of language (http://freakonomics.com/podcast/why-dont-we-speak-language/)
% and Mr. Bliss (http://www.radiolab.org/story/257194-man-became-bliss/)


%%%%%%%%%%%%%%%%%%%%%%%%%%%%%%%%%%%%%%%%%%%%%%%%%%%
%%%%%%%%%%%%%%%%%%%%%%%%%%%%%%%%%%%%%%%%%%%%%%%%%%%
% End of Section

\section{Writing and Pronunciation}

Basic pronunciation is covered in \emph{The Complete Lojban Language} \cite[Chapter 2, p.~16]{CLL}

A more thorough examination of Lojban phonetics can be found in the \emph{The Complete Lojban Language} \cite[Chapter 3, p.~34]{CLL}

\subsection{Orthography}

The conventions for writing a language, such as rules about spelling, capitalization, and punctuation, constitutes a language's \textbf{orthography}, or "rules for writing". \footnote{For more about Lojban orthography, see \emph{The Complete Lojban Language} \cite[Chapter 3, p.~33]{CLL} }

\subsubsection{Phonemic Orthography}


In Lojban, all sounds of the spoken language correspond to one, unambiguous written equivalent. This means that consonants and vowels don't have multiple possible pronunciations, unlike English. 

As an example of how different sounds in English can take different written forms, let's explore the word \emph{fish}. In English you can find the sound of \textbf{f} in \emph{fish} in the letters \textbf{gh} in \emph{enough} and \emph{tough}. The sound made by the \textbf{i} in \emph{fish} can also be found as \textbf{o} in w\textbf{o}men. Finally, the sound made by the letters \textbf{sh} in \emph{fish} can be found as \textbf{ti} in na\textbf{ti}on and mo\textbf{ti}on. In the end you should, logically, be able to represent the word \emph{fish} just as well in English with the re-spelling \emph{ghoti}\footnote{\emph{Ghoti} dates back to, at the latest, 1855. Wikipedia has an \href{https://en.wikipedia.org/wiki/Ghoti}{article} on this topic for further reading.} Of course this is absurd, but so are the rules of English pronunciation!

Lojban avoids this ambiguity by allowing only one possible letter or set of letters for each possible sound. 


%%%%%%%%%%%%%%%%%%%%%%%%%%%%%%%%%%%%%%%%%%%%%%%%%%%
% End of Subsection


\subsection{Vowels}

Vowels are pronounced in ways that are likely already familiar to you.

\begin{description}
\item[ ] \textbf{a} as in f\textbf{a}ther, \emph{mlatu} -- cat
\item[ ] \textbf{e} as in p\textbf{e}t, \emph{gerku} -- dog
\item[ ] \textbf{i} as in mach\textbf{i}ne, \emph{mi} -- me
\item[ ] \textbf{o} as in cl\textbf{o}ver, \emph{voksa} -- voice
\item[ ] \textbf{u} as in r\textbf{u}de, \emph{sutra} -- fast
\item[ ] \textbf{y} as in the \textbf{a} in \textbf{a}bout or \textbf{a}round 
\end{description}

%%%%%%%%%%%%%%%%%%%%%%%%%%%%%%%%%%%%%%%%%%%%%%%%%%%
% End of Subsection

\subsection{Dipthongs}

Dipthongs are a combination of two vowel sounds that collapse together to make one single syllable. 

\begin{description}
\item[ ] \textbf{ai} as in \textbf{ai}sle
\item[ ] \textbf{ei} as in the \textbf{ey} in th\textbf{ey}
\item[ ] \textbf{oi} as in the \textbf{oy} in t\textbf{oy}
\item[ ] \textbf{au} as in the \textbf{ow} in c\textbf{ow}
\end{description}

%%%%%%%%%%%%%%%%%%%%%%%%%%%%%%%%%%%%%%%%%%%%%%%%%%%
% End of Subsection

\subsubsection{Dipthongs Starting with I}

In Lojban, \textbf{i} often takes on a quality like the English \textbf{y}. For example, in English \textbf{y} before a vowel often forms a dipthong, exemplified by words like \emph{yes}, \emph{yarrow}, and \emph{yonder}. As a rule, in Lojban, when \textbf{i} comes before a vowel, it has the same effect, for example \emph{.iu} in Lojban is pronounced like the English word \emph{you}. 

\begin{description}
\item[ ] \textbf{ia} as in \textbf{ya}wn 
\item[ ] \textbf{ie} as in \textbf{ye}s 
\item[ ] \textbf{ii} as in \textbf{yi}eld 
\item[ ] \textbf{io} as in \textbf{yo}-\textbf{yo}
\item[ ] \textbf{iu} as in \textbf{you} 
\end{description}

%%%%%%%%%%%%%%%%%%%%%%%%%%%%%%%%%%%%%%%%%%%%%%%%%%%
% End of Subsection

\subsubsection{Dipthongs Starting with U}

Similar to the rules about \textbf{i} dipthongs, vowels occuring after \textbf{u} behave as \textbf{w} in English, e.g. \textbf{wa}ter, \textbf{We}dnesday, \textbf{wi}sh. 


\begin{description}
\item[ ] \textbf{ua} as in \textbf{wa}ter 
\item[ ] \textbf{ue} as in \textbf{we}st 
\item[ ] \textbf{ui} as in \textbf{we} 
\item[ ] \textbf{uo} as in \textbf{wo}ah or \textbf{Wo}din
\item[ ] \textbf{uu} as in \textbf{wu} or \textbf{woo} 
\end{description}

%%%%%%%%%%%%%%%%%%%%%%%%%%%%%%%%%%%%%%%%%%%%%%%%%%%
% End of Subsection

\subsection{Consonants}

There are a few irregular consonants in Lojban.
Lojban only pronounces \textbf{c} like the \textbf{sh} sound in English. 

\begin{description}
\item[ ] \textbf{c} as in fa\textbf{c}ial, \emph{citka} -- to eat
\item[ ] \textbf{j} as in the \textbf{s} in plea\textbf{s}ure, \emph{jitfa} -- false
\item[ ] \textbf{g} as in \textbf{g}ift, \emph{gunka} -- work
\item[ ] \textbf{x} as in \textbf{ch} in lo\textbf{ch}, \emph{xruti} -- to return, \emph{fraxu} -- to forgive
\end{description}



%%%%%%%%%%%%%%%%%%%%%%%%%%%%%%%%%%%%%%%%%%%%%%%%%%%
%%%%%%%%%%%%%%%%%%%%%%%%%%%%%%%%%%%%%%%%%%%%%%%%%%%
%%%%%%%%%%%%%%%%%%%%%%%%%%%%%%%%%%%%%%%%%%%%%%%%%%%
% End of Chapter

\mainmatter


% ... 




\appendix

\chapter{Glossary}

Languages are represented by their ISO 639-1 codes. See the Abbreviations chapter of the appendix for a list of the codes used in this book. \footnote{For a full list of ISO 639-1 codes, see \href{https://en.wikipedia.org/wiki/List_of_ISO_639-1_codes}{this Wikipedia article}}

The entry of the glossary has:

\begin{enumerate}
\item The word you are looking up in bold (e.g. \textbf{mlatu})
\item Its gloss equivalent in italics (e.g. \textit{cat}) 
\item The "etymology" for the Lojban word. Each natural language word is represented in the Lojban alphabet, e.g. the word cat would be written \emph{kat} according to how it would sound and be written in Lojban. The language code comes before each natural language word, indicating what language the word is from. The phonetic parts that were used from the natural-language word, to create the Lojban word, are in bold. For example, from the English word ``c\textbf{at}'', only ``\textbf{at}'' was used in ``ml\textbf{at}u''. The full example you would see in the entry for \emph{mlatu} would be: 

(en: k\textbf{at}, zh: \textbf{mau}, hi: bi\textbf{la}r, es: g\textbf{at}, ar: k\textbf{at})
\item The full definition of the word, e.g. x\textsubscript{1} is a cat [/ puss / pussy / kitten] [feline animal] of species / breed x\textsubscript{2}
\end{enumerate}


In the Lojban to English section, the entry for \emph{cat} would look like this:


\begin{quote}
\textbf{mlatu}, \textit{cat} (en: k\textbf{at}, zh: \textbf{mau}, hi: bi\textbf{la}r, es: g\textbf{at}, ar: k\textbf{at}) -- x\textsubscript{1} is a cat [/ puss / pussy / kitten] [feline animal] of species / breed x\textsubscript{2}
\end{quote}


\section{Lojban to English Vocabulary}

...

\subsection{A}

\begin{description}
\item[ ]
\end{description}

\subsection{B}

\begin{description}
\item[ ]
\end{description}

\subsection{C}

\begin{description}
\item[ ]
\end{description}

\subsection{D}

\begin{description}
\item[ ]
\end{description}

\subsection{E}

\begin{description}
\item[ ]
\end{description}

\subsection{F}

\begin{description}
\item[ ]
\end{description}

\subsection{G}

\begin{description}
\item[ ]
\end{description}

\subsection{H}

\begin{description}
\item[ ]
\end{description}

\subsection{I}

\begin{description}
\item[ ]
\end{description}

\subsection{J}

\begin{description}
\item[ ]
\end{description}

\subsection{K}

\begin{description}
\item[ ]
\end{description}

\subsection{L}

\begin{description}
\item[ ]
\end{description}

\subsection{M}

\begin{description}
\item[ ] \textbf{mlatu}, \textit{cat} (en: k\textbf{at}, zh: \textbf{mau}, hi: bi\textbf{la}r, es: g\textbf{at}, ar: k\textbf{at}) -- x\textsubscript{1} is a cat [/ puss / pussy / kitten] [feline animal] of species / breed x\textsubscript{2}
\end{description}

\subsection{N}

\begin{description}
\item[ ]
\end{description}

\subsection{O}

\begin{description}
\item[ ]
\end{description}

\subsection{P}


\textbf{plise}, \textit{apple} (en: a\textbf{pl}, zh: \textbf{pi}ngu\textbf{e}, hi: \textbf{se}b) -- x\textsubscript{1} is an apple [fruit] of species / strain x\textsubscript{2}.

\subsection{Q}

\begin{description}
\item[ ]
\end{description}

\subsection{R}

\begin{description}
\item[ ]
\end{description}

\subsection{S}

\begin{description}
\item[ ]
\end{description}

\subsection{T}

\begin{description}
\item[ ]
\end{description}

\subsection{U}

\begin{description}
\item[ ]
\end{description}

\subsection{V}

\begin{description}
\item[ ]
\end{description}

\subsection{W}

\begin{description}
\item[ ]
\end{description}

\subsection{X}

\begin{description}
\item[ ]
\end{description}

\subsection{Y}

\begin{description}
\item[ ]
\end{description}

\subsection{Z}

\begin{description}
\item[ ]
\end{description}


%%%%%%%%%%%%%%%%%%%%%%%%%%%%%%%%%%%%%%
% End of Section: Lojban to English Vocabulary 

\section{English to Lojban Vocabulary}

The English to Lojban section only provides the Lojban word for the equivalent glossed word in English. For more information about that word, look it up in the Lojban to English vocabulary section.

\subsection{A}

\begin{description}
\item[ ] \textbf{apple}, \textit{plise} 
\end{description}

\subsection{B}

\begin{description}
\item[ ]
\end{description}

\subsection{C}

\begin{description}
\item[ ] \textbf{cat}, \textit{mlatu}
\item[ ] \textbf{correct}, \textit{drani} 
\end{description}

\subsection{D}

\begin{description}
\item[ ]
\end{description}

\subsection{E}

\begin{description}
\item[ ]
\end{description}

\subsection{F}

\begin{description}
\item[ ]
\end{description}

\subsection{G}

\begin{description}
\item[ ]
\end{description}

\subsection{H}

\begin{description}
\item[ ]
\end{description}

\subsection{I}

\begin{description}
\item[ ]
\end{description}

\subsection{J}

\begin{description}
\item[ ]
\end{description}

\subsection{K}

\begin{description}
\item[ ]
\end{description}

\subsection{L}

\begin{description}
\item[ ]
\end{description}

\subsection{M}

\begin{description}
\item[ ]
\end{description}

\subsection{N}

\begin{description}
\item[ ]
\end{description}

\subsection{O}

\begin{description}
\item[ ]
\end{description}

\subsection{P}

\begin{description}
\item[ ]
\end{description}

\subsection{Q}

\begin{description}
\item[ ]
\end{description}

\subsection{R}

\begin{description}
\item[ ]
\end{description}

\subsection{S}

\begin{description}
\item[ ]
\end{description}

\subsection{T}

\begin{description}
\item[ ]
\end{description}

\subsection{U}

\begin{description}
\item[ ]
\end{description}

\subsection{V}

\begin{description}
\item[ ]
\end{description}

\subsection{W}

\begin{description}
\item[ ]
\end{description}

\subsection{X}

\begin{description}
\item[ ]
\end{description}

\subsection{Y}

\begin{description}
\item[ ]
\end{description}

\subsection{Z}

\begin{description}
\item[ ]
\end{description}


%%%%%%%%%%%%%%%%%%%%%%%%%%%%%%%%%%%%%%
% End of Section: English to Lojban Vocabulary 



%%%%%%%%%%%%%%%%%%%%%%%%%%%%%%%%%%%%%%%%%%%%%%%%%%%%%%%%%%%%%%%%%%%%%%%%%%%%%%%
%%%%%%%%%%%%%%%%%%%%%%%%%%%%%%%%%%%%%%%%%%%%%%%%%%%%%%%%%%%%%%%%%%%%%%%%%%%%%%%
% End of Glossary

\chapter{Abbreviations}

\section{ISO 639-1 Codes}

These are codes used throughout this book.

\begin{description}
\item[ ] Arabic: \textbf{ar}
\item[ ] Chinese: \textbf{zh}
\item[ ] English: \textbf{en}
\item[ ] Hindi: \textbf{hi}
\item[ ] Russian: \textbf{ru}
\item[ ] Spanish: \textbf{es}
\end{description}

\begin{thebibliography}{9}
\bibitem{CLL}
	John Woldemar Cowan
	\textit{The Complete Lojban Language},
	Logical Language Group,
	Version 1.1,
	2016-04-12,
	\href{https://lojban.github.io/cll/}{lojban.github.io}.
\bibitem{WheelocksLatin}
	Frederic M. Wheelock,
	\textit{Wheelock's Latin},
	HarperCollins Publishers, New York,
	6th edition, revised,
	2005.
\bibitem{PinkerWindow}
	Steven Pinker,
	\textit{Linguistics as a Window to Understanding the Brain},
	The Floating University,
	2011,
	\href{https://www.youtube.com/watch?v=Q-B_ONJIEcE}{www.youtube.com}.
\bibitem{BrownSci}
	James Cooke Brown,
	\textit{Loglan},
	Scientific American,
	Volume 202, Number 6, pp. 53 - 63,
	1960,
	\href{http://members.home.nl/w.dijkhuis/loglan_jcb/Brown_JC_loglan.html}{members.home.nl}.
\end{thebibliography}

\backmatter 

\chapter{Last Note}

\end{document}
